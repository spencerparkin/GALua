\documentclass[12pt]{article}

\usepackage{amsmath}
\usepackage{amssymb}
\usepackage{amsthm}
\usepackage{graphicx}
\usepackage{float}

\title{CGAUtil Math}
\author{Spencer T. Parkin}

\newcommand{\G}{\mathbb{G}}
\newcommand{\V}{\mathbb{V}}
\newcommand{\R}{\mathbb{R}}
\newcommand{\B}{\mathbb{B}}
\newcommand{\nvao}{o}
\newcommand{\nvai}{\infty}

\newtheorem{theorem}{Theorem}[section]
\newtheorem{definition}{Definition}[section]
\newtheorem{corollary}{Corollary}[section]
\newtheorem{identity}{Identity}[section]
\newtheorem{lemma}{Lemma}[section]
\newtheorem{result}{Result}[section]

\begin{document}
\maketitle

This document is formal documentation for the CGAUtil Lua module,
supplementary to the GALua module.  Here in is explained the
mathematics used by the CGAUtil module to compose and decompose
each geometric primitive of 3-dimensional CGA, the conformal model of geometric algebra
for 3-dimensional euclidean space.  It is assumed the reader is already
familiar with CGA.  In this document we let the outer product take precedence
over the inner product, and the geometric product take precedence over
the inner and outer products.

It should be noted that the CGA system is patented by David Hestenes, Hongbo Li, et. al.

\section{Composition}

We begin with an explanation of the composition of each geometric primitive.
The geometric primitives of CGA are listed in Table~\ref{table_geo_char}, along with
the parameters characterizing each geometry.\footnote{The reader may notice
that tangent points and free blades are missing from Table~\ref{table_geo_char}.
As points are simply degenerate spheres, tangent points arrise as the degenerate
point-pairs and circles.  Free blades are the intersection of distinct and parallel planes.
Tangent points and free blades are, as of this writing, not explicitly supported, but only implicitly
in the form of these special cases.}
\begin{table}
\begin{center}
\begin{tabular}{|c|c|c|c|c|c|}
\hline
 & weight & center & normal & radius & real/imaginary \\
\hline
point & yes & yes & no & no & no \\
\hline
flat-point & yes & yes & no & no & no \\
\hline
point-pair & yes & yes & yes & yes & yes \\
\hline
line & yes & yes & yes & no & no \\
\hline
circle & yes & yes & yes & yes & yes \\
\hline
plane & yes & yes & yes & no & no \\
\hline
sphere & yes & yes & no & yes & yes \\
\hline
\end{tabular}
\caption{The parameters characterizing each CGA geometry.}\label{table_geo_char}
\end{center}
\end{table}
Each geometry can be represented in a direct or dual form.
Table~\ref{tab_geo_dual_grades} lists each geometry again, this time giving
the grade of blade representing each geometrying in dual form.
\begin{table}
\begin{center}
\begin{tabular}{|c|c|c|}
\hline
grade 1 & grade 2 & grade 3 \\
\hline
point & circle & point-pair \\
\hline
sphere & line & flat-point \\
\hline
plane & & \\
\hline
\end{tabular}
\caption{CGAUtil's grades for dual geometries.}\label{tab_geo_dual_grades}
\end{center}
\end{table}
This table is not a comprehensive listing for each grade.
For example, the null pseudo-vectors of CGA are also dually
representative of points.  The table does, however, completely
describe the CGAUtil module's choice of which grades of blades it uses
to dually represent the geometric primitives of CGA.
The grade $g$ of a blade used by CGAUtil to directly represent each of the
geometric primitives in this table is given by $5-g$.

The composition (and decomposition) of each CGA geometry is given in terms of its
dual form, because a such form naturally presents the parameters (the columns of
Table~\ref{table_geo_char}) characterising each geometry (a row of Table~\ref{table_geo_char}).

For the compositions (and decompositions) of this document, we will let the variables $w$,
$c$, $n$ and $r$ denote the weight, center, normal and radius of each geometry, respectively.
Each of $w$ and $r$ are scalars while each of $c$ and $n$ are euclidean vectors.  The normal vector
$n$ is always of unit-length.  We always require $r\geq 0$.
Then, letting $\{e_1,e_2,e_2,\nvao,\nvai\}$ be a set of basis vectors generating the
vector space $\V$ that in turn generates the geometric algebra $\G$, and
letting $I$ denote the unit-pseudo scalar of $G$ and $i$ denote the
unit-pseudo scalar of the largest euclidean geometric sub-algebra of $G$,
we can now procede to give the composition of each geometry.

\subsection{Point Composition}

A point $\rho$ being characterized by a weight $w$ and center $c$, is given by
\begin{equation}\label{equ_point}
\rho = w\left(\nvao + c + \frac{1}{2}c^2\nvai\right).
\end{equation}

\subsection{Sphere Composition}

A sphere $\sigma$ being characterized by a weight $w$, center $c$ and radius $r$,
is given by
\begin{equation}\label{equ_sphere}
\sigma = w\left(\nvao + c + \frac{1}{2}(c^2-sr^2)\nvai\right),
\end{equation}
where $s$ is a scalar being $1$ or $-1$.  We have $s=1$ in the
case that $\sigma$ is a real sphere, and $s=-1$ in the case that
$\sigma$ is an imaginary sphere.

\subsection{Plane Composition}

A plane $\pi$ being characterized by a weight $w$, center $c$ and normal $n$, is
given by
\begin{equation}\label{equ_plane}
\pi = w\left(n + (c\cdot n)\nvai\right).
\end{equation}

\subsection{Circle Composition}

A circle $\gamma$ being characgterized by a weight $w$, center $c$, normal $n$
and radius $r$, is given by
\begin{align}
\gamma &=w\left(\nvao+c+\frac{1}{2}(c^2-sr^2)\nvai\right)\wedge(n+(c\cdot n)\nvai) \\
 &= w\left(\nvao\wedge n+(c\cdot n)\nvao\wedge\nvai+c\wedge n+
\left((c\cdot n)c - \frac{1}{2}(c^2-sr^2)n\right)\wedge\nvai\right),\label{equ_circle}
\end{align}
where $s$ is a scalar being $1$ or $-1$.  We have $s=1$ in the
case that $\gamma$ is a real circle, and $s=-1$ in the case that
$\gamma$ is an imaginary circle.  Our choice to multiply the sphere
left of the plane in the outer product is arbitrary, but a choice
warrenting documentation as this affects the sign of $w$.

\subsection{Line Composition}

A line $\lambda$ being characterized by a weight $w$, center $c$ and normal $n$, is
given by
\begin{equation}\label{equ_line}
\lambda = w\left(n + (c\wedge n)\nvai\right)i.
\end{equation}

\subsection{Point-Pair Composition}

A point-pair $\beta$ being characterized by a weight $w$, center $c$, normal $n$
and radius $r$, is given by
\begin{align}
\beta &= w\left(\nvao + c + \frac{1}{2}(c^2-sr^2)\nvai\right)\wedge(n+(c\wedge n)\nvai)i \\
 &= w\left(\nvao\wedge n+c\wedge n\wedge\nvao\wedge\nvai + c\cdot n -
\left((c\cdot n)c-\frac{1}{2}(c^2+sr^2)n\right)\wedge\nvai\right)i,\label{equ_pointpair}
\end{align}
where $s$ is a scalar being $1$ or $-1$.  We have $s=1$ in the case
that $\beta$ is a real point-pair, and $s=-1$ in the case that $\beta$ is an imaginary piont-pair.
Our choice to multiply the sphere left of the line in the otuer product is arbitrary, but a
choice warrenting documentation as this affects the sign of $w$.

\subsection{Flat-Point Composition}

A flat-point $\phi$ being characterized by a weight $w$ and center $c$, is given
by
\begin{align}\label{equ_flatpoint}
\phi &= w(n+(c\cdot n)\nvai)\wedge(n+(c\wedge n)\nvai)i \\
 &= w(1-c\wedge\nvai)i,
\end{align}
where here the unit-length normal $n$ cancels.  Our choice to multiply the plane
left of the line is arbitrary, but a choice warrenting documentation as this affects the sign of $w$.

\section{Decomposition}

Decomposition begins with a treatment of identification.  In CGA, it is possible
to show that the blades dually representative of real non-degenerate rounds are also
directly representative of imaginary rounds.  Given such a blade, CGAUtil resolves
the ambiguity in its identification as a specific geometric primitive of CGA by always
choosing the dual interpretation.  CGAUtil, therefore, only provides decomposition methods
that interpret blades as being dually representive of CGA geometries.  An attempt
to decompose the geometry directly reppresented by a given blade can then be
performed by simply applying a CGAUtil decomposition method to one of its duals.

A given decomposition method attempts to decompose a given blade
under the assumption that
it is of a specific dual form (i.e., the dual forms found in previous section on composition.)
Such an attempt will either pass or fail.
The decomposition methods are designed to fail if and only if the given blade
simply does not represent the assumed geomtric type.  It follows that the identification of
what geometries are dually represented by a given blade are found by simplying
attempting to apply, in any order, the set of all decomposition methods approriate
to the grade of that blade.  A convenience routine has been provided by CGAUtil
that performs this analysis on a given blade.  More than one geometric type
may be identified in certain cases (e.g., in the case of a degenerate sphere, this
is also a point.)

\subsection{Point Decomposition}

A point $\rho$, given in equation \eqref{equ_point}, may be decomposed as follows.
\begin{align}
w &= -\rho\cdot\nvai \\
c &= \nvao\wedge\nvai\cdot\frac{\rho}{w}\wedge\nvao\wedge\nvai
\end{align}
If the weight $w$ is zero, the decomposition fails.
Furthermore, to insure a correct decomposition, the validity of the following equation should be checked.
\begin{equation}
-2\nvao\cdot\frac{\rho}{w}=c^2
\end{equation}
The decomposition also fails if this equation is not satisified.

\subsection{Sphere Decomposition}

A sphere $\sigma$, given in equation \eqref{equ_sphere}, may be decomposed as follows.
\begin{align}
w &= -\sigma\cdot\nvai \\
c &= \nvao\wedge\nvai\cdot\frac{\sigma}{w}\wedge\nvao\wedge\nvai
\end{align}
If the weight $w$ is zero, the decomposition fails.
Realizing that $s^{-1}=s$, the radius $r$ may be found as
\begin{equation}
r^2 = s\left(c^2+2\nvao\cdot\frac{\sigma}{w}\right),
\end{equation}
where here we may choose $s=1$ or $s=-1$ so that $r^2\geq 0$.

\subsection{Plane Decomposition}

A plane $\pi$, given in equation \eqref{equ_plane}, may be decomposed as follows.
\begin{align}
w &= |\nvao\cdot\pi\wedge\nvai| \\
n &= \nvao\cdot\frac{\pi}{w}\wedge\nvai
\end{align}
If the weight $w$ is zero, the decomposition fails.

Then, assuming $c\wedge n=0$, we may write
\begin{equation}
c=-\left(\nvao\cdot\frac{\pi}{w}\right)n.
\end{equation}
Such an assumption is necessary, because the center of a plane is arbitrary
and not recoverable from composition.  Our choice of center $c$ here
gives us a position on the plane closest to the origin.

Notice that the sign of the weight $w$ is also lost in composition, and therefore
unrecoverable in decomposition.  The weight calculated here always satisfies
the condition $w>0$.

Also notice that the results of the above decomposition steps are considered
undefined in the case that $\pi\cdot\nvai\neq 0$.  For this reason, before
any steps are taken, we fail under this condition.

\subsection{Circle Decomposition}

A circle $\gamma$, given in equation \eqref{equ_circle}, may be decomposed as follows.
\begin{align}
w &= |\nvao\wedge\nvai\cdot\gamma\wedge\nvai| \\
n &= -\nvao\wedge\nvai\cdot\frac{\gamma}{w}\wedge\nvai
\end{align}
If the weight $w$ is zero, the decomposition fails.
Like the plane, here, the weight $w$ and normal $n$ are recovered
only up to sign.  That we are able to recover the center $c$ follows
from cancellation of sines.
\begin{equation}
c = -n\left(\nvao\wedge\nvai\cdot
\frac{\gamma}{w}\wedge\nvao\nvai\right).
\end{equation}
Realizing that $s^{-1}=s$, the radius $r$ may be found as
\begin{equation}
r^2=s\left(c^2+2\left(\nvao\wedge\nvai\cdot\nvao\wedge\frac{\gamma}{w}-
(c\cdot n)c\right)n\right),
\end{equation}
where here we may choose $s=1$ or $s=-1$ so that $r^2\geq 0$.

\subsection{Line Decomposition}

A line $\lambda$, given in equation \eqref{equ_line}, may be decomposed
as follows.
\begin{align}
w &= |(\nvao\cdot\lambda\wedge\nvai)i| \\
n &= \left|\left(\nvao\cdot\frac{\lambda}{w}\wedge\nvai\right)i\right|
\end{align}
If the weight $w$ is zero, the decomposition fails.

Then, assuming $c\cdot n=0$, we may write
\begin{equation}
c = \left(\left(\nvao\cdot\frac{\lambda}{w}\right)n\right)i.
\end{equation}
Like planes, the center $c$ is lost in composition.  Our choice
of $c$ here is a position on the line closest to the origin.

Also like the plane, here our decomposition is undefined,
and therefore fails, in the case that $\lambda i\cdot\nvai\neq 0$.

\subsection{Point-Pair Decomposition}

A point-pair $\beta$, given in equation \eqref{equ_pointpair}, may be
decomposed as follows.
\begin{align}
w &= |(\nvao\wedge\nvai\cdot\beta\wedge\nvai)i| \\
n &= \left(\nvao\wedge\nvai\cdot\frac{\beta}{w}\wedge\nvai\right)i
\end{align}
The decomposition fails if the weight $w$ is zero.

Like the circle, cancellation of sines allows us to recover the center $c$ as
\begin{equation}
c = -n\left(\nvao\wedge\nvai\cdot\frac{\beta}{w}\wedge\nvao\nvai\right)i.
\end{equation}
Realizing that $s^{-1}=s$, the radius $r$ may be found as
\begin{equation}
r^2 = s\left(-c^2 + 2\left(\left(\nvao\wedge\nvai\cdot\nvao\wedge\frac{\beta}{w}\right)i+(c\cdot n)c\right)n\right),
\end{equation}
where here we may choose $s=1$ or $s=-1$ so that $r^2\geq 0$.

\subsection{Flat-Point Decomposition}

A flat-point $\phi$, given in equation \eqref{equ_flatpoint}, may be decomposed as follows.
\begin{align}
w &= -(\nvao\cdot\phi\wedge\nvai)i \\
c &= \left(\nvao\cdot\frac{\phi}{w}\right)i
\end{align}

\end{document}