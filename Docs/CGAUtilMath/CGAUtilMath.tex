\documentclass[12pt]{article}

\usepackage{amsmath}
\usepackage{amssymb}
\usepackage{amsthm}
\usepackage{graphicx}
\usepackage{float}

\title{CGAUtil Math}
\author{Spencer T. Parkin}

\newcommand{\G}{\mathbb{G}}
\newcommand{\V}{\mathbb{V}}
\newcommand{\R}{\mathbb{R}}
\newcommand{\B}{\mathbb{B}}
\newcommand{\nvao}{o}
\newcommand{\nvai}{\infty}

\newtheorem{theorem}{Theorem}[section]
\newtheorem{definition}{Definition}[section]
\newtheorem{corollary}{Corollary}[section]
\newtheorem{identity}{Identity}[section]
\newtheorem{lemma}{Lemma}[section]
\newtheorem{result}{Result}[section]

\begin{document}
\maketitle

This document is formal documentation for the CGAUtil Lua module,
supplementary to the GALua module.  Here in is explained the
mathematics used by the CGAUtil module to compose and decompose
each geometric primitive of 3-dimensional CGA, the conformal model of geometric algebra
for 3-dimensional euclidean space.  It is assumed the reader is already
familiar with CGA.  In this document we let the outer product take precedence
over the inner product, and the geometric product take precedence over
the inner and outer products.

It should be noted that the CGA system is patented by David Hestenes, Hongbo Li, et. al.

\section{Composition}

We begin with an explanation of the composition of each geometric primitive.
The geometric primitives of CGA are listed in Table~\ref{table_geo_char}, along with
the parameters characterizing each geometry.\footnote{The reader may notice
that tangent points and free blades are missing from Table~\ref{table_geo_char}.
As points are simply degenerate spheres, tangent points arrise as the degenerate
point-pairs and circles.  Free blades are the intersection of distinct and parallel planes.
Tangent points and free blades are, as of this writing, not explicitly supported, but only implicitly
in the form of these special cases.}
\begin{table}
\begin{center}
\begin{tabular}{|c|c|c|c|c|c|}
\hline
 & weight & center & normal & radius & real/imaginary \\
\hline
point & yes & yes & no & no & no \\
\hline
flat-point & yes & yes & no & no & no \\
\hline
point-pair & yes & yes & yes & yes & yes \\
\hline
line & yes & yes & yes & no & no \\
\hline
circle & yes & yes & yes & yes & yes \\
\hline
plane & yes & yes & yes & no & no \\
\hline
sphere & yes & yes & no & yes & yes \\
\hline
\end{tabular}
\caption{The parameters characterizing each CGA geometry.}\label{table_geo_char}
\end{center}
\end{table}
Each geometry can be represented in a direct or dual form.
Table~\ref{tab_geo_dual_grades} lists each geometry again, this time giving
the grade of blade representing each geometrying in dual form.
\begin{table}
\begin{center}
\begin{tabular}{|c|c|c|}
\hline
grade 1 & grade 2 & grade 3 \\
\hline
point & circle & point-pair \\
\hline
sphere & line & flat-point \\
\hline
plane & & \\
\hline
\end{tabular}
\caption{CGAUtil's grades for dual geometries.}\label{tab_geo_dual_grades}
\end{center}
\end{table}
This table is not a comprehensive listing for each grade.
For example, the null pseudo-vectors of CGA are also dually
representative of points.  The table does, however, completely
describe the CGAUtil module's choice of which grades of blades it uses
to dually represent the geometric primitives of CGA.
The grade $g$ of a blade used by CGAUtil to directly represent each of the
geometric primitives in this table is given by $5-g$.

The composition (and decomposition) of each CGA geometry is given in terms of its
dual form, because a such form naturally presents the parameters (the columns of
Table~\ref{table_geo_char}) characterising each geometry (a row of Table~\ref{table_geo_char}).

For the compositions (and decompositions) of this document, we will let the variables $w$,
$c$, $n$ and $r$ denote the weight, center, normal and radius of each geometry, respectively.
Each of $w$ and $r$ are scalars while each of $c$ and $n$ are euclidean vectors.  The normal vector
$n$ is always of unit-length.  We always require $r\geq 0$.
Then, letting $\{e_1,e_2,e_2,\nvao,\nvai\}$ be a set of basis vectors generating the
vector space $\V$ that in turn generates the geometric algebra $\G$, and
letting $I$ denote the unit-pseudo scalar of $G$ and $i$ denote the
unit-pseudo scalar of the largest euclidean geometric sub-algebra of $G$,
we can now procede to give the composition of each geometry.

\subsection{Point Composition}

A point $\rho$ being characterized by a weight $w$ and center $c$, is given by
\begin{equation}\label{equ_point}
\rho = w\left(\nvao + c + \frac{1}{2}c^2\nvai\right).
\end{equation}

\subsection{Sphere Composition}

A sphere $\sigma$ being characterized by a weight $w$, center $c$ and radius $r$,
is given by
\begin{equation}\label{equ_sphere}
\sigma = w\left(\nvao + c + \frac{1}{2}(c^2-sr^2)\nvai\right),
\end{equation}
where $s$ is a scalar being $1$ or $-1$.  We have $s=1$ in the
case that $\sigma$ is a real sphere, and $s=-1$ in the case that
$\sigma$ is an imaginary sphere.

\subsection{Plane Composition}

A plane $\pi$ being characterized by a weight $w$, center $c$ and normal $n$, is
given by
\begin{equation}\label{equ_plane}
\pi = w\left(n + (c\cdot n)\nvai\right).
\end{equation}

\subsection{Circle Composition}

A circle $\gamma$ being characgterized by a weight $w$, center $c$, normal $n$
and radius $r$, is given by
\begin{align}\label{equ_circle}
\gamma &=w\left(\nvao+c+\frac{1}{2}(c^2-sr^2)\nvai\right)\wedge(n+(c\cdot n)\nvai) \\
 &= w\left(\nvao\wedge n+(c\cdot n)\nvao\wedge\nvai+c\wedge n+
\left((c\cdot n)c - \frac{1}{2}(c^2-sr^2)n\right)\wedge\nvai\right),
\end{align}
where $s$ is a scalar being $1$ or $-1$.  We have $s=1$ in the
case that $\gamma$ is a real circle, and $s=-1$ in the case that
$\gamma$ is an imaginary circle.  Our choice to multiply the sphere
left of the plane in the outer product is arbitrary, but a choice
warrenting documentation as this affects the sign of $w$.

\subsection{Line Composition}

A line $\lambda$ being characterized by a weight $w$, center $c$ and normal $n$, is
given by
\begin{equation}\label{equ_line}
\lambda = w\left(n + (c\wedge n)\nvai\right)i.
\end{equation}

\subsection{Point-Pair Composition}

A point-pair $\beta$ being characterized by a weight $w$, center $c$, normal $n$
and radius $r$, is given by
\begin{align}\label{equ_piontpair}
\beta &= w\left(\nvao + c + \frac{1}{2}(c^2-sr^2)\nvai\right)\wedge(n+(c\wedge n)\nvai)i \\
 &= w\left(\nvao\wedge n+c\wedge n\wedge\nvao\wedge\nvai + c\cdot n +
\left((c\cdot n)c-\frac{1}{2}(c^2+sr^2)n\right)\wedge\nvai\right)i,
\end{align}
where $s$ is a scalar being $1$ or $-1$.  We have $s=1$ in the case
that $\beta$ is a real point-pair, and $s=-1$ in the case that $\beta$ is an imaginary piont-pair.
Our choice to multiply the sphere left of the line in the otuer product is arbitrary, but a
choice warrenting documentation as this affects the sign of $w$.

\subsection{Flat-Point Composition}

A flat-point $\phi$ being characterized by a weight $w$ and center $c$, is given
by
\begin{align}\label{equ_flatpoint}
\phi &= w(n+(c\cdot n)\nvai)\wedge(n+(c\wedge n)\nvai)i \\
 &= w(1-c\wedge\nvai)i,
\end{align}
where here the unit-length normal $n$ cancels.  Our choice to multiply the plane
left of the line is arbitrary, but a choice warrenting documentation as this affects the sign of $w$.

\section{Decomposition}

Decomposition begins with a treatment of identification.  In CGA, it is possible
to show that the blades dually representative of real non-degenerate rounds are also
directly representative of imaginary rounds.  Given such a blade, we cannot
uniquely determine a method of decomposition.  Between the two possible
geometric interpretations of the blade, we must choose which one we want,
and then apply the corresponding decomposition method.
There may be other instances were such a
choice needs to be made.  Any decomposition method of CGAUtil, therefore, does not attempt to identify
blades as being of any particular type of CGA geometry.  Instead, the decomposition
methods of CGAUtil attempt to decompose a given blade under the assumption
that it is of the a specific geometric type.  Such an attempt will either pass or fail.
An attempt has been made, however, in a provided routine, to identify
a given blade as being one of a number returned geometric types.  The user
can then use this information to select a decomposition method.

% To identify a CGA geometry, why not just run it through all decomposers,
% and see which ones pass and which ones fail?  There's probably a faster
% way to identify blades, but it's a good start, and must be correct.

\subsection{Point Decomposition}

\subsection{Sphere Decomposition}

\subsection{Plane Decomposition}

\subsection{Circle Decomposition}

\subsection{Line Decomposition}

\subsection{Point-Pair Decomposition}

\subsection{Flat-Point Decomposition}

\end{document}